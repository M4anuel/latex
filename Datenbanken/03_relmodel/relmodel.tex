\documentclass[../DatenbankenFS23.tex]{subfiles}

\begin{document}
\section{Das Relationenmodell}
\begin{problem}
    Wir nehmen an, wir wollen die Daten zu einer von Menge von Autos in
    unserer Datenbank halten.
    Für jedes Auto soll dessen \textbf{Marke} und \textbf{Farbe} abgespeichert werden.
\end{problem}
\subsection{Attribute \& Domäne}
    Wir können jedes Auto als Paar $(Marke, Farbe)$ darstellen.
    Eine Menge von Autos entspricht somit einer Menge von solchen Paaren.
    Das heisst wir können eine Menge von Autos durch eine 2-stellige Relation
    repräsentieren.
    \[Autos := \{(Opel, silber), (VW, rot), (Audi, schwarz)\} .\]
    Wir können diese Relation als Tabelle mit zwei Spalten darstellen:
    \newline
    \begin{figure}[ht]
        \centering
            $\begin{matrix}
                Autos & \\
                \hline
                Marke & Farbe \\
                \hline
                Opel & silber \\
                VW & rot \\
                Audi & schwarz \\
                \hline
            \end{matrix}$
    \end{figure} \newline
    wobei 
    \begin{itemize}
        \item[] \textbf{Attribut} Name der Spalte $(Marke, Farbe)$
        \item[] \textbf{Domäne} Mögliche Werte, die ein Attribut annehmen kann
    \end{itemize}

\subsection{Null Werte}
    $Null$ wird verwendet, wenn der Wert eines Attributs unbekannt ist,
    \begin{itemize}
        \item[] weil wir den Wert nicht kennen oder
        \item[] oder, weil das Attribut keinen Wert besitzt.
    \end{itemize}

\subsection*{Konvention}
\begin{itemize}
    \item[] $Null$ gehört zu jeder Domäne.
    \item[] in der Tabellenform schreiben wir häufig − anstelle von $Null$.
\end{itemize}    

\begin{bemerkung}
    \textbf{Semantik von Null} \newline
    Da wir $Null$ verwenden, um auszudrücken, dass der Wert eines Attributs
    unbekannt ist, erhält Null eine spezielle Semantik bezüglich der Gleichheit:
    \[Es\ gilt\ \textbf{nicht},\ dass\ Null = Null.\ (1)\]
    Wenn wir zwei unbekannte Werte vergleichen, so wissen wir eben nicht, ob
    sie gleich sind. Deshalb verwenden wir eine Semantik für die $(1)$ der Fall ist.
    \newline
\end{bemerkung}

\begin{defn}[Schwache Gleichheitsrelation]
    An einigen Stellen werden wir zwei $Null$ Werte als gleichwertig betrachten
    müssen. Dazu führen wir folgende schwache Gleichheitsrelation zwischen
    zwei atomaren Objekten $a$ und $b$ ein:
    \[a \simeq b\ g.d.w.\ a = b\ oder\ (a\ ist\ Null\ und\ b\ ist\ Null).\]
    Für zwei n-Tupel definieren wir analog:
    \[ (a_1, \dots , a_n) \simeq (b_1, \dots , bn)\ g.d.w.\ a_i \simeq b_i\ f\ddot{u}r\ alle\ 1 \leq i \leq n.\]
\end{defn}
\begin{beispiel}
    Seien $a$, $b$, und $c$ paarweise verschiedene Objekte, die verschieden von $Null$ sind.

    \begin{itemize}
        \item $a \simeq a$
        \item $a \not\simeq a$
        \item $a \not\simeq Null$
        \item $Null \simeq Null$
        \item $(a, b) \simeq (a, b)$
        \item $(a, Null, c) \simeq (a, Null, c)$
        \item $(a, Null, c) \not\simeq (a, b, c)$
    \end{itemize}
\end{beispiel}

\subsection{Relationsschema \& Relationale Datenbank}
Relationenschemata (oder einfach nur Schemata) spezifizieren die bei
Relationen verwendeten Attribute und Domänen. Es handelt sich dabei um
Sequenzen der Form
\[(A_1 : D_1, \dots , A_n : D_n) ,\]
wobei $A_1, \dots , A_n$ Attribute mit den jeweiligen Domänen $D1, \dots , Dn$ sind.

\begin{beispiel}
    $(Marke : text, Baujahr : integer, Farbe : color_enum, Fahrer : text)$
\end{beispiel}


\subsection*{Konvention}
\begin{itemize}
    \item Wir schreiben manchmal 
    \[(A_1, \dots , A_n)\ anstelle\ von\ (A_1 : D_1, \dots , A_n : D_n),\]
    wenn sich die Domänen unmittelbar aus dem Kontext ergeben oder
    unwichtig sind.
    \item Wir sagen $R$ ist eine Relation über $A_1, \dots , A_n$, wenn $R$ eine Relation
    über den dazugehörenden Domänen $D_1, \dots , D_n$ ist.
    \item Wir sagen dann auch $R$ ist eine Instanz des Schemas $(A_1, \dots , A_n)$.
\end{itemize}

\begin{bemerkung}
    Als \emph{relationales Datenbank-Schema} (oder kurz \emph{DB-Schema}) bezeichnen wir
    die Menge aller verwendeten Relationenschemata.
\end{bemerkung}
\begin{defn}[Relationale Datenbank]
    Als \emph{relationale Datenbank} (oder kurz \emph{relationale DB}) bezeichnen wir das
    verwendete relationale Datenbank-Schema zusammen mit den momentanen
    Werten der Relationen. \newline
    Eine relationale Datenbank besteht somit aus
    \begin{itemize}
        \item einem DB-Schema und
        \item den aktuellen Instanzen aller Schemata des DB-Schemas.
    \end{itemize}
    Wir sprechen in diesem Zusammenhang auch von einer \emph{Instanz} eines
    DB-Schemas und meinen damit die Menge der aktellen Instanzen aller
    Schemata des DB-Schemas. 
\end{defn}

\begin{defn}[Projektion auf Attribute]
    Es seien $A_1, \dots , A_n$ Attribute,
    \[S = (A_1, \dots , A_n)\]
    ein Relationenschema und $R$ eine Instanz von $S$.
    
    \begin{itemize}
        \item 
         Ist $t$ ein $n$-Tupel, das zu $R$ gehört, so schreiben wir $t[A_i]$ für den Wert
        von $t$ bei Attribut $A_i$. Für $(a_1, \dots , a_i, \dots , a_n) \in R$ heisst das
        $(a_1, \dots , a_i , \dots , a_n)[A_i] = a_u$.    
        \item Ist $K = (A_{i_1}, \dots , A_{i_m})$ eine Sequenz von Attributen, so definieren wir
        für ein $n$-Tupel $(a_1, \dots , a_i, \dots , a_n) \in R  $
        $(a_1, \dots , a_i, \dots , a_n)[K] := (a_{i_1}, \dots , a_{i_m})$.
    \end{itemize}
\end{defn}

\begin{beispiel}
    Betrachten wir die Relation\newline
        $\begin{matrix}
            \textbf{Autos}\\
            \hline
            \textbf{Marke} & \textbf{Farbe} & \textbf{Baujahr} & \textbf{Vorname} & \textbf{Nachname}\\
            \hline
            Opel & silber & 2010 & Tom & Studer\\
            Opel & schwarz & 2010 & Eva & Studer\\
            VW & rot & 2014 & Eva Studer\\
            Audi & schwarz & 2014 & Eva Meier\\
            \hline
        \end{matrix}$
            \newline\newline
    Es sei nun
\[t := (Opel, schwarz, 2010, Eva, Studer) .\]
Damit gilt
\[t[(Marke, Farbe)] = (Opel, schwarz)\]
und
\[t[(Nachname, Baujahr)] = (Studer, 2010) .\]
\end{beispiel}

\begin{bemerkung}
    Für $s, t \in R$ bedeutet $s[K] = t[K]$, dass die Werte von $s$ und $t$ in allen
    Attributen aus $K$ übereinstimmen.
\end{bemerkung}

\subsection{Schlüssel}
\begin{problem}
    Wie können wir in einer Instanz $R$ von $S$ die einzelnen Elemente unterscheiden?
\end{problem}
Dazu wählen wir einen sogenannten \emph{Primärschlüssel}. Dies ist eine Sequenz
von Attributen
\[K = (A_{i_1}, \dots , A_{i_m}) .\]
Dann verlangen wir für alle Instanzen $R$ von $S$ und alle $s, t \in R$, dass
$s[K] = t[K] \Rightarrow s \simeq t .$

\subsection*{Konvention}
Wir geben den Primärschlüssel an, indem wir beim Relationenschema
diejenigen Attribute unterstreichen, welche zum gewählten Primärschlüssel
gehören.

\begin{beispiel}
    Betrachten wir die Relation\newline
    $\begin{matrix}
        \textbf{Autos}\\
        \hline
        \textbf{Marke} & \textbf{Farbe} & \textbf{Baujahr} & \textbf{Vorname} & \textbf{Nachname}\\
        \hline
        Opel & silber & 2010 & Tom & Studer\\
        Opel & schwarz & 2010 & Eva & Studer\\
        VW & rot & 2014 & Eva  & Studer\\
        Audi & schwarz & 2014 & Eva  & Meier\\
        \hline
    \end{matrix}$\newline\newline
    Es sind bspw. die beiden folgenden Primärschlüssel möglich:
    \[(Marke, Farbe)\ und\ (Baujahr, Vorname, Nachname) . (3)\]
    Falls wir $(Marke, Farbe)$ als Primärschlüssel wählen, so geben wir das
    Schema wie folgt an:
    \[(\underline{Marke}, \underline{Farbe}, Baujahr, Vorname, Nachname) .\]
\end{beispiel}

\subsection*{Sprechende Schlüssel}
In einer echten Datenbankanwendung sind wahrscheinlich beide möglichen
Primärschlüssel aus (3) ungeeignet.
Es ist nämlich gut möglich, dass wir später dieser Relation weitere Autos
hinzufügen möchten. Da kann es dann sein, dass ein zweiter roter VW
eingefügt werden soll oder ein weiteres Auto mit Baujahr 2010 und dem
Fahrer Tom Studer.\newline
In der Praxis wird oft ein zusätzliches Attribut, z.B. $AutoId$, hinzugefügt,
welches als Primärschlüssel dient.
Dieses Attribut hat nur den Zweck, die verschiedenen Elemente der
Relation eindeutig zu bestimmen. Es beschreibt aber keine echte
Eigenschaft von Autos. Wir nennen einen solchen Primärschlüssel einen
\emph{nicht-sprechenden} Schlüssel.\newline
Ein \emph{sprechender} Schlüssel hingegen hat eine logische Beziehung zu einem
oder mehreren Attributen des Schemas.\newline
Es ist gute Praxis \emph{keine} sprechenden Schlüssel zu verwenden, da diese die
Tendenz haben zu zerbrechen. \newline
Das heisst, früher oder später wird eine Situation auftreten, in der ein
neues Tupel eingefügt werden soll, dessen sprechender Schlüssel bereits ein
Tupel in der Relation bezeichnet. Oder es kann sein, dass das System der
Schlüsselgenerierung komplett geändert wird. Beispiele dazu sind:
\begin{itemize}
    \item das System der AHV-Nummern (eindeutige Personennummer der
    Alters- und Hinterlassenenversicherung) in der Schweiz, welches 2008
    geändert wurde,
    \item die Internationale Standardbuchnummer (ISBN), für die 2005 ein
    revidierter Standard eingeführt wurde.
\end{itemize}

\begin{problem}
    \begin{center}
    $\begin{matrix}
        \textbf{Autos}\\
        \hline
        \textbf{Marke} & \textbf{Farbe} & \textbf{Baujahr} & \textbf{Vorname} & \textbf{Nachname}\\
        \hline
        Opel & silber & 2010 & Tom & Studer\\
        Opel & schwarz & 2010 & Eva & Studer\\
        VW & rot & 2014 & Eva  & Studer\\
        Audi & schwarz & 2014 & Eva  & Meier\\
        \hline
    \end{matrix}$\newline
\end{center}
    In dieser Tabelle sind die Daten zu Eva Studer doppelt abgespeichert.
\end{problem}
\subsection*{Besser: zwei Zabellen}
\begin{figure}[ht]
    \centering
    $\begin{matrix}
        \textbf{Autos}\\
        \hline
        \underline{\textbf{Marke}} & \underline{\textbf{Farbe}} & \textbf{Baujahr} & \textbf{FahrerId}\\
        \hline
        Opel & silber & 2010 & 1\\
        Opel & schwarz & 2010 & 2\\
        VW & rot & 2014 & 2\\
        Audi & schwarz & 2014 & 3\\
        \hline
    \end{matrix}$
\end{figure}
\begin{figure}[ht]
    \centering
    $\begin{matrix}
        \textbf{Personen}\\
        \hline
        \underline{\textbf{PersId}} & \textbf{Vorname} & \textbf{Nachname}\\
        \hline
        1 & Tom & Studer\\
        2 & Eva & Studer\\
        3 & Eva  & Meier\\
        \hline
    \end{matrix}$
\end{figure} \newline
    FahrerId \emph{referenziert} die Tabelle Personen \newline
    FahrerId ist ein \emph{Fremdschlüssel} für Personen
FOLIE 25
\end{document}