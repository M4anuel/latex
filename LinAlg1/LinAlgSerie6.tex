\documentclass{article}
% common packages used in maths
\usepackage{amsmath}
\usepackage{amsthm}%possible problem
\usepackage{amssymb}
\usepackage{amsfonts}
\author{
Pascal Zürcher 22-111-314 & 
Leandro Lüthi 22-105-035 & 
Manuel Flückiger 22-112-502}
\title{Lineare Algebra Serie 6 Übungsaufgaben}
\date{\today}

\begin{document}
\maketitle
\section{Basis und Dimensionen verschiedener Vektorräume bestimmen.}
    \begin{enumerate}

        \item[a)]
        Da $x_1 = 2x_2$ sind $x_1$ und $x_2$ linear 
        unabhängig. da $x_1 + x_4 = x_3$ gilt, ist $x_3$ von $x_1$ und $x_4$ 
        abhängig. Somit sind nur $x_1 und x_4$ linear unabhängig.
        Damit ist $dim = 2$ und eine mögliche Basis ist 
        \newline
        $<x_1, x_4>$ ,
        $<\left(\begin{array}{c}2\\1\\2\\0\end{array}\right)$,
        $\left(\begin{array}{c}2\\1\\3\\1\end{array}\right)>$

        \item[b)]
        $\alpha x_1 + \beta x_2 + \gamma x_3 + \delta x_4 = 0$ wenn $\alpha = \beta=\delta=0$
        gelten müsste, um diese Gleichung zu erfüllen, wären $x_1,x_2,x_3,x_4$ linear unabhängig.
        Daraus folgt, dass nicht alle linear unabhängig sind. $x_1+3x_2+x_3+2x_4 = 0$ Da es zur 
        Darstellung einer der Vektoren immer alle drei Anderen benötigt (siehe (*)), sind 
        drei Vektoren linear unabhängig. Somit ist dim = 3 und eine mögliche Basis 
        ist basis $<x_1,x_3,x_4>$
        \newline
        \fbox{\parbox{\linewidth}{
        \begin{itemize}
            \item[]*
            \item[] $x_1 = -3x_2-x_3-2x_4$
            \item[] $x_2 = -\frac{1}{5}x_1-x_3-\frac{2}{3}x_4$
            \item[] $x_3 = -x_1-3x_2-2x_4$
            \item[] $x_4 = -\frac{1}{2}x_1-\frac{3}{2}x_2-\frac{1}{2}x_3$
        \end{itemize}}}

        \item[c)]
        $Mat(2,3,\mathbb{R})$ ist isomorph zu $\mathbb{R}^6$, da jedes 
        Elemente der Matrix irgend ein Element aus $\mathbb{R}$ ist.
        Somit ist dim = 6 und eine Basis zum Beispiel:
        \newline
        basis$<\left<
        \left(\begin{array}{cc}1&0\\0&0\\0&0\end{array}\right),
        \left(\begin{array}{cc}0&0\\1&0\\0&0\end{array}\right),
        \left(\begin{array}{cc}0&0\\0&0\\1&0\end{array}\right),
        \left(\begin{array}{cc}0&1\\0&0\\0&0\end{array}\right),
        \left(\begin{array}{cc}0&0\\0&1\\0&0\end{array}\right),
        \left(\begin{array}{cc}0&0\\0&0\\0&1\end{array}\right),\right>$

        \item[d)]
        $\left\{\left(\begin{array}{cc}\alpha&\beta\\\beta&\gamma\end{array}\right)
        :\alpha,\beta,\gamma\in\mathbb{R}\right\}\subseteq Mat(2,2,\mathbb{R})$
        \newline
        $a_{12}$ und $a_{21}$ müssen den gleichen Wert haben, also:
        \newline
        $\alpha\left(\begin{array}{cc}1&0\\0&0\end{array}\right)+
        \beta\left(\begin{array}{cc}0&1\\1&0\end{array}\right)+
        \gamma\left(\begin{array}{cc}0&0\\0&1\end{array}\right)$
        \newline
        Die drei Matrizen bilden die Basis mit drei Variablen $\Rightarrow$ dim = 3. 
        \newline
        Basis: basis$\left<\left(\begin{array}{cc}1&0\\0&0\end{array}\right),
        \left(\begin{array}{cc}0&1\\1&0\end{array}\right),
        \left(\begin{array}{cc}0&0\\0&1\end{array}\right)\right>$

        \item[e)]
        Span$\{1+x, x+x^2, x^2+x^3,1+x^2,x+x^3\}\subseteq$ Pol$\mathbb{R}$
        \newline
        U=Span$\left(
            \left(\begin{array}{c}1\\1\\0\\0\end{array}\right)
            \left(\begin{array}{c}0\\1\\1\\0\end{array}\right)
            \left(\begin{array}{c}0\\0\\1\\1\end{array}\right)
            \left(\begin{array}{c}1\\0\\1\\0\end{array}\right)
            \left(\begin{array}{c}0\\1\\0\\1\end{array}\right)
        \right)
        \newline
        \left( \begin{array}{ccccccc}
            1&0&0&1&0&|&0\\
            1&1&0&0&1&|&0\\
            0&1&1&1&0&|&0\\
            0&0&1&0&1&|&0\\
        \end{array}\right)
        \rightarrow
        \left( \begin{array}{ccccccc}
            1&0&0&1&0&|&0\\
            0&1&0&-1&1&|&0\\
            0&1&1&1&0&|&0\\
            0&0&1&0&1&|&0\\
        \end{array}\right)
        \rightarrow
        \left( \begin{array}{ccccccc}
            1&0&0&1&0&|&0\\
            0&1&0&-1&1&|&0\\
            0&0&1&2&-1&|&0\\
            0&0&1&0&1&|&0\\
        \end{array}\right)
        \rightarrow
        \left( \begin{array}{ccccccc}
            1&0&0&1&0&|&0\\
            0&1&0&-1&1&|&0\\
            0&0&1&2&-1&|&0\\
            0&0&0&-2&0&|&0\\
        \end{array}\right)
        \rightarrow
        \left( \begin{array}{ccccccc}
            1&0&0&0&1&|&0\\
            0&1&0&0&1&|&0\\
            0&0&1&0&-1&|&0\\
            0&0&0&-2&0&|&0\\
        \end{array}\right)$
        \newline
        dim = 4, da vier Pivot-Vektoren existieren und eine 
        Basis ist: basis<($1$),($x$),($x^2$),($-2x^3$),(die Pivot-Vektoren)

        \item[f)]
        $\{p(x)\in $Pol$_s \mathbb{R}: p(1)=0\}$
        \newline
        $p(x)=a_0+a_1x+a_2x^2+a_3x^3$
        \newline
        $p(1)=0\Rightarrow a_0=-a_1-a_2-a_3 (a_0+a_11+a_21^2+a_31^3=0)$
        \Rightarrow Basis ist $\{(-1+x),(-x^2),(-1+x^3)\}:$ 3 Komponenten $\Rightarrow$ dim = 3

        \item[g)]
        fuck
    \end{enumerate}
    \section{}
\end{document}