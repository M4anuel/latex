\documentclass[../main.tex]{subfiles}

\begin{document}
\chapter{Abbildungen?}\label{chp:real}
\section{Erinnerung}
\begin{definition}[Erinnerung an letztes mal]
  Eine Abblidung $f:A \rightarrow B$ heisst:
  \begin{enumerate}
    \item [i)] injektiv: falls $\forall\ a_1, a_2 \in A$,
     $a_1 \neq a_2$, gilt $f(a_1) \neq f(a_2)$.
    \item [ii)] surjektiv: falls $\forall\ b \in B$ existiert
    ein $a \in A$ mit $f(a) = b$
    \item [ii)] bijektiv: falls $f$ injektiv und surjektiv ist. 
    $\Rightarrow f$ hat einen Umkehrwert $f^{-1} \ B \rightarrow A$.
  \end{enumerate}
\end{definition}
\begin{example}
  Sei $A = B = \mathbb{N} = {0,1,2,3,...}$
  \begin{enumerate}
    \item $f \colon \mathbb N &\to \mathbb N \\
    n &\mapsto 2n.$
    \item[i)] $f: \mathbb{N} \rightarrow \mathbb{N}$ injektiv
    $n \mapsto 2n$ ¬surjektiv
    
  \end{enumerate}
\end{example}

Hier ist
\[\mathbb N = \{0, 1, 2, 3, 4, \dots\}\]
die Menge der Natürlichen Zahlen.

\begin{theorem*}[Euklid]
  Die Seite und Diagonale eines ebenen Quadrats sind nicht kommensurabel.
\end{theorem*}

\begin{proof}
  Dieser Beweis ist geometrisch, nach Euklid. Wir nehmen an, es gäbe $L > 0$ und
  $m,n \in \mathbb N$ mit $x = mL$ und $d = nL$. Wir zeigen, dass das zu einem Widerspruch
  führt.
  Wir stellen fest, dass die Längen $x_{1} = d-x$ und $d_{1} = 2x - d$
  ebenfalls die Seite und Diagonale eines Quadrats bilden, siehe
  Abbildung~\ref{fig:euklid}.

  \begin{figure}[htb]
    \centering
    \begin{minipage}{0.4\linewidth}
      \centering
      \includegraphics{images/addieren}
    \end{minipage}%
  \end{figure}

  Weiterhin gilt, dass sowohl $x_{1}$ als auch $d_{1}$ ganze Vielfache von $L$ sind:
  \begin{align*}
    x_{1} = d-x = (n-m)L, \\
    d_{1} = 2x-d = (2m -n)L.
  \end{align*}
  Nach Pythagoras gilt $d^{2} = 2x^{2}$, und somit $d \leq 3/2\cdot x$, da ${(3/2)}^{2} > 2$.
  Daraus folgt, dass
  \(x_{1} = d - x \leq 1/2 \cdot x\).
  Iteriere dieses Verfahren und erhalte eine Serie von Quadraten mit Seiten
  $x_{2}, x_{3}, \dots$ und Diagonalen $d_{2}, d_{3}, \dots$. Es gilt
  \(x_{k} \leq 1/2^k \cdot x\).
  Ausserdem ist jedes $x_{k}$ (und $d_{k}$) ein ganzes Vielfaches von $L$.
  Wähle nun $k$ so gross, dass
   \(x_{k} \leq 1/2^{k}\cdot x < L\).
   Dies, zusammen mit dem Fakt, dass
   $x_{k}$ ein ganzes Vielfaches von $L$ ist,
   impliziert, dass $x_{k} = 0$, was unmöglich ist. Deshalb können $x$ und $d$
   nicht kommensurabel sein.
\end{proof}

Wir haben diese Aussage mit einem sogenannten \emph{Widerspruchsbeweis} bewiesen.
Hierfür haben wir eine Annahme getroffen, und diese zu einem Widerspruch geführt.
Dies zeigt, dass unsere Annahme falsch war.

\subsection*{Zeitgenössische Umformulierung}
Seien $a,b > 0$ zwei kommensurable Längen. Das heisst, es existieren $L > 0$
und $m,n \in \mathbb N$ mit $a = mL$, $b= nL$. Dann gilt
\[\frac{a}{b} = \frac{mL}{nL} = \frac{m}{n},\]
das heisst, das Verhältnis $a/b$ ist eine \emph{rationale Zahl}.
Zurück zum Quadrat mit Seite $x$ und Diagonale $d$. Nach Pythagoras
gilt $d^{2} = 2x^{2}$. Falls $x=mL$ und $d=nL$ gilt,
dann also
\[2 = \frac{d^{2}}{x^{2}} = {\left( \frac{d}{x}\right)}^{2} = {\left(\frac{n}{m}\right)}^{2},\]
und somit
\(
2m^{2} = n^{2}\).
Die linke Seite dieser Gleichung ist durch $2$ teilbar. Dies impliziert, dass $n^{2}$, und
somit auch $n$, durch $2$ teilbar ist. Schreibe nun $n = 2k$. Schreibe $n = 2k$ mit
$k \in \mathbb N$. Setze das in die Gleichung $m^2 = n^2$ ein und erhalte
\[
2m^{2} = {(2k)}^{2} = 4k^{2},
\]
beziehungsweise
\(m^{2} = 2k^{2}\).
Die rechte Seite ist durch $2$ teilbar, also auch $m$.
Wir schliessen, dass sowohl $n$ als auch $m$ durch $2$ teilbar sind.
Schreibe noch $m = 2\ell$ mit $\ell \in \mathbb N$. Es gilt also
\[ 2 = {\left(\frac{n}{m}\right)}^{2}
  = {\left(\frac{k}{\ell}\right)}^{2}.\]
In anderen Worten sind Zähler und Nenner beide gerade.
Iteriere dieses Verfahren $k$ mal, bis $n/2^{k} < 1$. Dann entsteht ein Widerspruch.

\begin{corollary*}
  Die Gleichung $z^{2} = 2$ hat keine rationale Lösung, das heisst,
  keine Lösung der Form $z = p/q$ mit $p, q \in \mathbb N$ und $q > 0$.
\end{corollary*}

\subsubsection*{Der goldene Schnitt}
Wir werden nun ein weiteres Beispiel
einer irrationalen Zahl nach Euklid untersuchen.
\end{document}