\documentclass{article}
\usepackage{lisitings}
\usepackage{enumitem}

\documentclass[a4paper,12pt]{article}
% -- Loading the code block package:
\usepackage{listings}
% -- Basic formatting
\usepackage[utf8]{inputenc}
\usepackage[english]{babel}
\usepackage{times}
\setlength{\parindent}{8pt}
\usepackage{indentfirst}
% -- Defining colors:
\usepackage[dvipsnames]{xcolor}
\definecolor{codegreen}{rgb}{0.706, 0.824, 0.451}
\definecolor{codegray}{rgb}{0.475, 0.475, 0.475}
\definecolor{codepurple}{rgb}{0.62, 0.525, 0.784}
\definecolor{backcolour}{rgb}{0.18, 0.18, 0.18}
\definecolor{pink}{rgb}{0.69, 0.322, 0.475}
\definecolor{white}{rgb}{0.839, 0.839, 0.839}
% Definig a custom style:
\lstdefinestyle{mystyle}{
    backgroundcolor=\color{backgroundcolor},   
    commentstyle=\color{codegray}, 
    keywordstyle=\color{pink},
    numberstyle=\tiny\color{codegray},
    stringstyle=\color{codegreen},
    basicstyle=\ttfamily\footnotesize\bfseries\color{white},
    breakatwhitespace=false,         
    breaklines=false,                 
    captionpos=t,                    
    keepspaces=true,                 
    numbers=left,                    
    numbersep=5pt,                  
    showspaces=false,                
    showstringspaces=false,
    showtabs=false,                  
    tabsize=2
}
% -- Setting up the custom style:
\lstset{style=mystyle}


\author{Abdihakin Sahal Omar 20-947-107 & Manuel Flückiger 22-11-502}
\title{P1 Serie 7}
\date{\today}

\begin{document}
\maketitle

\textbf{Excercise 1}
\begin{enumerate}[label=\alph*]
    \item True
    \item False
    \item False
    \item True
    \item True
    \item False
\end{enumerate}

\textbf{Excercise 2}
\begin{verbatim}
for i = 1 to 100 
    if (i is divisible by 3 AND i is divisible by 5) 
        output "BizzBuzz" 
    else if (i is divisible by 3) 
        output "Bizz" 
    else if (i is divisible by 5) 
        output "Buzz" 
    else output i 
    end if 
end for
\end{verbatim}

\textbf{Excercise 3}
\begin{verbatim}
Input: Arrays A and B of length n 
Output: Array C of length n + 1 
Function BinarySum (A, B, n) 
    Let carry = 0 
    Let C[n+1] // Initialize output array 
    For i = 0 to n-1 
        C[i] = (A[i] + B[i] + carry) % 2 // Calculate sum of current bit 
        carry = (A[i] + B[i] + carry) / 2 // Calculate carry for next iteration 
    End For 
    C[n] = carry // Store carry in the n-th bit of C 
    Return C 
End Function
\end{verbatim}

\textbf{Excercise 4}
\begin{lstlisting}[language=Java]
public static <T> void shuffle(List<T> list) {
    int n = list.size() - 1;
    Random rand = new Random();
    for(int i = n; i > 0; --i) {
        int r = rand.nextInt(i+1);
        List<T> copy = new LinkedList<>(list);
        list.set(i, copy.get(r));
        list.set(r, copy.get(i));
    }
}         
public static void swap(List<Integer> list, int i, int r) {
    int temp = list.get(i); 
    list.set(i, list.get(r)); 
    list.set(r, temp); 
}
\end{lstlisting}
\textbf{Excercise 5}
\begin{verbatim}
    Java 1.0
    Python 1.1
    12345
\end{verbatim}
\textbf{Excercise 6} \newline
The stack s will look like this: s = [5, 72, 37]\newline
\textbf{Excercise 7}\newline
The stack s will look like this: s = [72, 37, 15]\newline
\textbf{Excercise 8}
\begin{lstlisting}[language=Java]
    public boolean set(int index, Object object) {
        if(index > size - 1)
            return false;
        else {
            listElements[index] = object;
            return true;
    }
}

\end{lstlisting}
\textbf{Excercise 9}
\begin{lstlisting}[language=Java]
    public int size() {
		if (this.startNode == null)
			return 0;
		else {
			int count = 1;
			Node<E> temp = this.startNode;
			while(temp.getNext() != null) {
				temp = temp.getNext();
				count++;
			}
			return count;
		}
    }
	
\end{lstlisting}
\textbf{Excercise 10}\newline
Die Java-collection ist ein abstrakter Typ, der die gemeinsamen Verhaltensweisen und Merkmale verschiedener Sammlungen definiert. 
Die Get- und Set-Methoden sind spezifisch für die Implementierung der Sammlung und werden daher nicht in der Schnittstelle angegeben. 
Jede Implementierung der Schnittstelle kann die get- und set-Methoden anders implementieren, so dass es dem Benutzer der Sammlung obliegt, 
die Implementierung der von ihm verwendeten Sammlung zu kennen und zu wissen, wie die get- und set-Methoden richtig zu verwenden sind.
Eine Liste hat eine bestimmte Reihenfolge. Ein Set hat keine, also ist set.get(i) kein sinnvolles Konzept.
Es wäre sinnvoll, eine get(i)-Operation für SortedSet und TreeSet zu haben, aber es gibt sie nicht.

\end{document}